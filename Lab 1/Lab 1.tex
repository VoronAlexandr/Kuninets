\documentclass[11pt]{article}
\usepackage{amsmath,amssymb,amsthm}
\usepackage{algorithm}
\usepackage[noend]{algpseudocode} 

%---enable russian----

\usepackage[utf8]{inputenc}
\usepackage[russian]{babel}


% PROBABILITY SYMBOLS
\newcommand*\PROB\Pr 
\DeclareMathOperator*{\EXPECT}{\mathbb{E}}
\usepackage{fancyhdr}

% Sets, Rngs, ets 
\newcommand{\N}{{{\mathbb N}}}
\newcommand{\Z}{{{\mathbb Z}}}
\newcommand{\R}{{{\mathbb R}}}
\newcommand{\Zp}{\ints_p} % Integers modulo p
\newcommand{\Zq}{\ints_q} % Integers modulo q
\newcommand{\Zn}{\ints_N} % Integers modulo N

% Landau 
\newcommand{\bigO}{\mathcal{O}}
\newcommand*{\OLandau}{\bigO}
\newcommand*{\WLandau}{\Omega}
\newcommand*{\xOLandau}{\widetilde{\OLandau}}
\newcommand*{\xWLandau}{\widetilde{\WLandau}}
\newcommand*{\TLandau}{\Theta}
\newcommand*{\xTLandau}{\widetilde{\TLandau}}
\newcommand{\smallo}{o} %technically, an omicron
\newcommand{\softO}{\widetilde{\bigO}}
\newcommand{\wLandau}{\omega}
\newcommand{\negl}{\mathrm{negl}} 

% Misc
\newcommand{\eps}{\varepsilon}
\newcommand{\inprod}[1]{\left\langle #1 \right\rangle}


\newcommand{\handout}[5]{
	\noindent
	\begin{center}
		\framebox{
			\vbox{
				\hbox to 5.78in { {\bf Научно-исследовательская практика} \hfill #2 }
				\vspace{4mm}
				\hbox to 5.78in { {\Large \hfill #5  \hfill} }
				\vspace{2mm}
				\hbox to 5.78in { {\em #3 \hfill #4} }
			}
		}
	\end{center}
	\vspace*{4mm}
}

\newcommand{\lecture}[4]{\handout{#1}{#2}{#3}{Scribe: #4}{#1}}

\newtheorem{theorem}{Теорема}
\newtheorem{lemma}{Лемма}
\newtheorem{definition}{Определение}
\newtheorem{corollary}{Следствие}
\newtheorem{fact}{Факт}

% 1-inch margins
\topmargin 0pt
\advance \topmargin by -\headheight
\advance \topmargin by -\headsep
\textheight 8.9in
\oddsidemargin 0pt
\evensidemargin \oddsidemargin
\marginparwidth 0.5in
\textwidth 6.5in

\parindent 0in
\parskip 1.5ex

\setcounter{page}{74}
\setlength{\parindent}{5ex}

\begin{document}
	\thispagestyle{fancy}
	\rhead{CHAP. 4}
	\chead{Теория сравнений}
	\lhead{\thepage}
	\lecture{Сравнения}{Лето 2020}{}{Кунинец Артем}
	Таким образом, можно обнаружить, что \[1! +2! +3! + 4!...+100! \equiv  1! +  +2! +3! + 0 + ...+ 0 \equiv 9 \pmod{12}\] 
	Из этого следует, что остаток от деления данной суммы на $12$ равен $9$.
	
	В последней теореме было доказано, что если  $a \equiv b \pmod{n}$ , то $ca = cb \pmod {n}$ для любого $c$. 
	Обратное, однако, будет неверно. Например $2\cdot4 \equiv 2\cdot1 \pmod{6}$, но в то же время ${4 \not\equiv 1 \pmod 6}$.
	Вкратце: нельзя безоговорочно отменять общий фактор арифметики сравнений.
	
	При соответствующих мерах предосторожности отмена допустима; Один важный шаг в этом 
	направлении обеспечивается следующей теоремой.
	
	
	\begin{theorem}
		\label{th4-3}
		Если $ca \equiv cb \pmod n$, то $a\equiv b \pmod {\frac{n}{d}}$, где $d = \text{НОД} (c,n)$
		
	\end{theorem}
	
	\begin{proof}
		Предположим, что \[c(a-b)=ca-cb=kn\]  для некоторого k. Зная, что $\text{НОД}(c,n) = d$, можно сделать вывод, что существуют простые числа $r$ и $s$, удовлетворяющие равенствам $c = dr, n = ds$. Когда эти значения подставляются в уравнение, $d$ сокращается и получается следующее: \[r(a-b)=ks\]
		
		Следовательно, $s\mid r(a-b)$ и $\text{НОД}(r,s) = 1$. Лемма Евклида подразумевает, что $s\mid a-b$, которое может быть переписано в виде $a \equiv b \pmod{n}$; другими словами, $a\equiv b \pmod {\frac{n}{d}}$
		
		
		
	\end{proof}
	
	Теорема~\ref{th4-3} наиболее полезна, когда добавляется условие: $\text{НОД}(c,n) = 1$, тогда сокращение может быть выполнено без изменения модуля.
	
	\begin{corollary}
		\label{col-1}
		Если $ca \equiv cb \pmod n$ и $\text{НОД}(c,n) = 1$, то  $a \equiv b \pmod n$
	\end{corollary}
	Мы воспользуемся моментом, чтобы записать особый случай следствия~\ref{col-1}, которым мы будем часто пользоваться, а именно:
	
	\begin{corollary}
		\label{col-2}
		Если $ca \equiv cb \pmod p$ и $p$ не делит $c$ ($p$ - простое число), то $a \equiv b \pmod p$
	\end{corollary}
	
	\begin{proof}
		Из условий $p$ не делит $c$ и $p$ - простое число следует, что $\text{НОД} (c,p) = 1$
	\end{proof}
	
	\section*{Пример 4--4}
	
	Рассмотрим сравнение $33\equiv 15 \pmod 9$ или $11\cdot3\equiv 3\cdot5 \pmod 9$. Т.к. $\text{НОД} (3,9) = 3$, то из теоремы~\ref{th4-3} следует, что $11\equiv 5 \pmod 3$. Ещё одним примером служит сравнение ${-35\equiv 45 \pmod 8}$, которое эквивалентно сравнению $5\cdot(-7)\equiv 5\cdot9 \pmod 9$. Целые числа $5$ и $8$, являются взаимно простыми, мы можем упростить, чтобы получить сравнение $-7\equiv 9 \pmod8$
	
	
	\thispagestyle{fancy}
	\rhead{\thepage}
	\chead{Основные свойства сравнений}
	\lhead{SEC. 4--2}
	
	Обратим внимание на тот факт, что в теореме~\ref{th4-3} нет необходимости указывать, что  ${c \not\equiv 0 \pmod n}$.
	Наоборот, если бы  $c \equiv 0 \pmod n$, то $\text{НОД} (c,n) = n$ и из теоремы бы следовало, что  $a\equiv b \pmod 1$, но, как мы сказали ранее, это является тривиальным случаем.
	
	Есть еще одна любопытная ситуация, которая может возникнуть с сравнениями: произведение двух целых чисел, ни одно из которых не совпадает с нулем, может оказаться равным нулю. Например, $4\cdot 3\equiv0 \pmod {12}$, но $4\not\equiv0 \pmod {12}$ и $3\not\equiv0 \pmod {12}$. Таким образом, если $ab\equiv0\pmod n$ и $\text{НОД} (a,n) = 1$, то $b\equiv0\pmod n$; следствие~\ref{col-1} позволяет нам на законных основаниях сократить множитель $а$ с обеих сторон сравнения
	$ab\equiv a\cdot0\pmod n$. Таким образом, если $ab\equiv0\pmod p$, где $p$ -- простое число, то либо $a\equiv0\pmod p$, либо $b\equiv0\pmod p$
	
	\section* {\center Задания 4.2}
	
	\begin{enumerate}
		\item Докажите следующие утверждения:
		\begin{enumerate}
			\item Если $a \equiv b \pmod n$ и  $m\mid n$ , то $a \equiv b \pmod m$
			\item Если $a \equiv b \pmod n$ и  $c>0$ , то $ca \equiv cb \pmod {cn}$
			\item Если $a \equiv b \pmod n$ и все целые числа $a,b,n$ делятся $d>0$, то на, то  ${a / d \equiv b / d \pmod {n / d}}$
		\end{enumerate}
		\item Приведите пример , чтобы показать, что $a^{2}\equiv b^{2} \pmod {	n}$ не означает, что $a\equiv b \pmod {n}$
		\item Если $a\equiv b \pmod {n}$, докажите, что $\text{НОД} (a,n) = \text{НОД} (b,n)$
		\item 
		\begin{enumerate}
			\item Найдите остатки от деления $2^{50}$ и $41^{65}$ на $7$
			\item Найдите остаток от деления на $4$ выражения \[1^{5}+2^{5}+...+99^{5}+100^{5}\] 
		\end{enumerate}
		\item Если $a_1, a_2 , ..., a_n$ полное кольцо вычетов по модулю $n$ и $\text{НОД} (a,n) = 1$, докажите, что 
		${aa_1,aa_2,...,aa_n}$ тоже кольцо вычетов по модулю n (Подсказка: достаточно показать, что рассматриваемые числа  совпадают по модулю $n$) 
		\item Докажите, что $0, 1, 2, 2^{2},...,2^{9}$ -- кольцо вычетов по модулю 11, а $0, 1^{2}, 2^{2},...,10^{2}$ -- нет
		\item Докажите следующие утверждения:
		\begin{enumerate}
			\item Если $НОД\text{(a,n)} = 1$, то числа \[c,c+a,c+2a,...,c+(n-1)a\] формируют кольцо вычетов по модулю $n$ (для любого $c$)
			\item Любые $n$ последовательных целых чисел формируют кольцо вычетов по модулю $n$(Подсказка: используйте пункт а)
			\item Произведение любого набора из $n$ последовательных целых чисел делится на $n$.
		\end{enumerate}
		\item Докажите, что если $a\equiv b \pmod {n_1}$ и $a\equiv b \pmod {n_2}$, то $a\equiv b \pmod {n}$, где $n = \text{НОК}(n_1,n_2)$. Следовательно, если $n_1$ и $n_2$ взаимно простые $a\equiv b \pmod {n_1\cdot n_2}$
		
		\thispagestyle{fancy}
		\rhead{\thepage}
		\chead{Признаки делимости}
		\lhead{SEC. 4--3}
		\item Приведите пример, демонстрирующий, что $a^{k}\equiv b^{k} \pmod {n}$ и $k\equiv j \pmod {n}$ не означает, что $a^{j}\equiv b^{j} \pmod {n}$
		\item Докажите следующие утверждения:
		\begin{enumerate}
			\item Если a -- это нечётное целое число, то $a^{2}\equiv 1\pmod {8} $
			\item Для любого целого числа $a$, $a^{3}\equiv$ 0,1 или 8$\pmod {8}$
			\item Для любого целого числа $a$, $a^{3}\equiv a\pmod {6}$
			\item Если целое число a не делится на $2$ и $3$, то $a^{2} \equiv 1\pmod {24}$
			\item Если a является и квадратом, и кубом, то  $a \equiv$ $0,1,9$ или $28 \pmod {36}$
		\end{enumerate}
		\item Установите, что если a нечетное целое число, то \[a^{2^{n}}\equiv 1\pmod {2^{n+2}}\] для любого $n \geq 1$ (Подсказка: Воспользуйтесь методом математической индукции)
		\item Используйте теорию сравнений, чтобы доказать, что \[8\mid 2^{44}-1 \quad 97\mid2^{48}-1\]
		\item Докажите, что если $ab\equiv cd \pmod {n}$ и $b\equiv d \pmod {n}$, в то время как $\text{НОД} (b,n) = 1$, то $a\equiv c \pmod {n}$
		\item Если $a\equiv b \pmod {n_1}$ и $a\equiv c \pmod {n_2}$, докажите, что $b\equiv c \pmod {n}$, где \\
		$n = \text{НОД} (n_1,n_2)$ ($n$ -- целое число)
	\end{enumerate}
	
	\section* {4.3 Признаки делимости}
	Одно из наиболее интересных применений теории конгруэнтности заключается в поиске специальных критериев, при которых данное целое число делится на другое целое число. В своей основе эти тесты делимости зависят от системы обозначений, используемой для присвоения «имен» целым числам, и, в частности, от факта, что число $10$ взято в качестве основы для нашей системы счисления. Начнем с того, что покажем, что при заданном целом $b>1$ любое положительное целое число $N$ может быть написано однозначно с точки зрения степеней $b$ как
	\[N = a_mb^{m} + a_{m-1}b^{m-1} + ... + a_{2}b^{2} + a_{1}b^{1} + a_{0},\] где коэф-ты $a_{k}$ могут принимать значения от $0$ до $b-1$. Алгоритм деления даёт целые числа $q_{1}$ и $a_{0}$, удовлетворяющие равенству 
	
	\[N = q_1b+a_0, \qquad 0 \leq a_{0}\leq b\]
	
	Если $q_{1}\geq b $, мы можем разделить ещё раз и получим:
	\[N = q_2b+a_1, \qquad 0 \leq a_{1}\leq b\]  
	
	\thispagestyle{fancy}
	\rhead{\thepage}
	\chead{Признаки делимости}
	\lhead{SEC. 4--3}
	Теперь подставим $q_1$ в исходное равенство, получим: \[N = (q_{2}b+a_{1})b + a_{0} = q_{2}b^{2} + a_{1}b^{1} + a_{0} .\]
	До тех пор, пока $q_{2}>b$, мы можем продолжать деление. Следующим шагом получим ${q_{2}=q_{3}b+a_{2}}$, где ${0\leq a_{2} \leq b}$   
	\[N = q_{3}b^{3} + a_{2}b^{2} + a_{1}b^{1} + a_{0}\]
	Т.к.  $N > q_{1} > q_{2} >...\geq 0$ строго убывающая числовая последовательность, этот процесс конечен; предположим, что на $(n-1)$ делении процесс заканчивается, где \[q_{m-1} = q_{m}b + a_{m-1}, \qquad \quad  0\geq a_{m-1}<b \]
	и $0 \geq q_{m} < b$. Приравняв $a_{m} = q_{m}$, мы получим: \[N = a_{m}b^{m} + a_{m-1}b^{m-1} + ... +a_{1}b^{1} + a_{0},\] что и было нашей целью
	
	Чтобы показать единственность данного разложения, предположим (от обратного), что N раскладывается двумя способами; скажем,\[N = a_{m}b^{m} + a_{m-1}b^{m-1} + ... +a_{1}b^{1} + a_{0} = c_{m}b^{m} + c_{m-1}b^{m-1} + ... +c_{1}b^{1} + c_{0},\] где $ 0 \geq a_{i} < b$ для любого $i$ и $0 \geq c_{j} < b$ для любого $j$ (мы можем использовать тоже самое m, добавляя члены с коэффициентами $a_{i} = 0$ или $c_{j} = 0$, если потребуется). Вычтем из первого разложение второе и получим: \[0 = d_{m}b^{m} + ... + d_{1}b^{1} + d_0,\]
	
\end{document}


	
	
